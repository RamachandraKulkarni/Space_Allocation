\documentclass[11pt,a4paper]{article}

\usepackage[utf8]{inputenc}
\usepackage[T1]{fontenc}
\usepackage{lmodern}
\usepackage[margin=1in]{geometry}
\usepackage{hyperref}
\usepackage{booktabs}
\usepackage{longtable}
\usepackage{enumitem}
\usepackage{listings}
\usepackage{xcolor}
\usepackage{parskip}

\hypersetup{
  colorlinks=true,
  linkcolor=blue!60!black,
  urlcolor=blue!60!black,
}

\lstset{
  basicstyle=\ttfamily\small,
  backgroundcolor=\color{gray!10},
  frame=single,
  rulecolor=\color{gray!40},
  breaklines=true,
  columns=fullflexible,
  keepspaces=true,
}

\title{\textbf{Space Allocation Planner}\\[0.5em]
  \large How-to Guide}
\author{Space Allocation Team}
\date{\today}

\begin{document}

\maketitle
\tableofcontents
\newpage

% ────────────────────────────────────────────────────────────────
\section{Overview}
% ────────────────────────────────────────────────────────────────

The Space Allocation Planner is a React + TypeScript web dashboard that helps
programme administrators and facilities teams answer one question:
\textbf{``Given N students across several programs, how do we assign studio
groups to available rooms without exceeding building capacity?''}

Key features:
\begin{itemize}
  \item Upload or swap room-inventory CSVs with no code changes needed.
  \item Create student groups (studios) with configurable size caps.
  \item Optionally mix students from different programs within each studio.
  \item A three-pass allocation algorithm that respects per-floor capacity
        buffers.
  \item Visual diagnostics that flag unassigned groups and exhausted floors.
  \item One-click CSV export of the final allocation.
  \item Built-in staffing cost modelling (faculty, TAs, graders).
\end{itemize}

% ────────────────────────────────────────────────────────────────
\section{Quick Start}
% ────────────────────────────────────────────────────────────────

\begin{lstlisting}[language=bash]
# 1. Install dependencies
npm install

# 2. Start the development server
npm run dev
# Opens at http://localhost:5173

# 3. Production build
npm run build

# 4. Preview the production build locally
npm run preview
\end{lstlisting}

% ════════════════════════════════════════════════════════════════
\part*{For Non-Technical Users}
\addcontentsline{toc}{section}{How-to Guide --- Non-Technical Users}
% ════════════════════════════════════════════════════════════════

\section{What the Application Does}

The Space Allocation Planner takes two inputs --- your \textbf{room inventory}
and your \textbf{programme enrolment numbers} --- and automatically assigns
student groups to rooms.  It ensures that no room or floor is over-booked
beyond a safe buffer, and it calculates associated staffing costs.

You do not need any programming knowledge to use the application once it is
running in your web browser.

% ────────────────────────────────────────────────────────────────
\section{Step-by-Step Usage}
% ────────────────────────────────────────────────────────────────

\subsection{Step 1 --- Open the Application}

Navigate to the URL where the app is hosted (e.g.,
\texttt{http://localhost:5173} when running locally).  You will see the
\textbf{Space Allocation Planner} dashboard with a form at the top.

\subsection{Step 2 --- Set Allocation Parameters}

\begin{longtable}{@{}p{4cm}p{10cm}@{}}
\toprule
\textbf{Field} & \textbf{What It Means} \\
\midrule
\endfirsthead
\toprule
\textbf{Field} & \textbf{What It Means} \\
\midrule
\endhead
Number of Programs &
  How many separate academic programs are being allocated (e.g., 2 for
  Architecture and Interior Design). \\
Studio Cap &
  The maximum number of students in a single studio group (e.g., 20). \\
Allow multi-program mixing &
  When checked, each studio group will contain an equal share of students
  from every program.  When unchecked, studios are single-program only. \\
Program Sizes &
  For each program, enter a label (e.g., ``Architecture'') and the number
  of enrolled students.  Use \textbf{Remove} to delete a row or increase
  the program count to add more. \\
\bottomrule
\end{longtable}

\subsection{Step 3 --- Set Financial Inputs (Optional)}

These fields feed the staffing cost model.  They do not affect room
assignments.

\begin{longtable}{@{}p{4.5cm}p{9.5cm}@{}}
\toprule
\textbf{Field} & \textbf{What It Means} \\
\midrule
\endfirsthead
\toprule
\textbf{Field} & \textbf{What It Means} \\
\midrule
\endhead
Total Students (override) &
  Leave blank to use the sum of program sizes.  Enter a number here only
  if you need the cost model to assume a different student count. \\
Semesters per Year &
  Typically 2.  Used to annualise TA compensation. \\
TA/FA Compensation &
  Dollar amount paid to each TA or FA per semester. \\
Faculty / TAs-FAs / Graders &
  Head counts for each staff role.  Adjust these to model different
  scenarios. \\
\bottomrule
\end{longtable}

\subsection{Step 4 --- Click ``Calculate Allocation''}

The dashboard updates with several result panels:

\begin{longtable}{@{}p{3.5cm}p{10.5cm}@{}}
\toprule
\textbf{Panel} & \textbf{What It Shows} \\
\midrule
\endfirsthead
\toprule
\textbf{Panel} & \textbf{What It Shows} \\
\midrule
\endhead
Totals Row &
  Summary counts: rooms available, floors, max capacity, students
  required. \\
Studios Table &
  Every studio group generated, its size, program breakdown, and the room
  it was assigned to. \\
Rooms Table &
  Each room, its base and dynamic capacity, extra capacity used, and the
  studios placed in it. \\
Finance Summary &
  Annualised staffing cost breakdown per role (compensation, ERE, risk,
  tech fee, admin charge). \\
Floor Limits &
  Per-floor capacity status showing how much buffer remains on each
  floor. \\
Diagnostics &
  Warnings and errors, such as groups that could not be placed or floors
  that ran out of buffer. \\
\bottomrule
\end{longtable}

\subsection{Step 5 --- Refine the Results}

\begin{description}[style=nextline]
  \item[Toggle Rooms]
    Use the \emph{Room Selection} panel to include or exclude individual
    rooms (or individual rooms within a combined zone) before
    recalculating.
  \item[Rotate Allocation]
    Click \emph{Rotate Allocation} to shuffle the order rooms are
    considered.  This produces a different valid assignment without
    changing the groups.
  \item[Export CSV]
    Click \emph{Export CSV} to download the final allocation as a
    spreadsheet-compatible file.
\end{description}

\subsection{Step 6 --- Reset}

Click \textbf{Reset defaults} at the bottom of the form to restore all
fields to their original values.

% ────────────────────────────────────────────────────────────────
\section{Understanding the Results}
% ────────────────────────────────────────────────────────────────

\begin{itemize}
  \item \textbf{Green rows} in the Studios table indicate successfully
        assigned groups.
  \item \textbf{Red or highlighted rows} indicate unassigned groups ---
        these studios could not fit in any available room.
  \item A \textbf{``No floor buffer left''} diagnostic means a floor has
        reached its maximum allowed capacity (base + 15\% buffer by
        default).
  \item \textbf{Extra Capacity Used} in the Rooms table shows how many
        seats beyond the room's base occupancy are being used, drawn from
        the floor buffer.
\end{itemize}

% ────────────────────────────────────────────────────────────────
\section{Updating Room Data (No Code Required)}
% ────────────────────────────────────────────────────────────────

The application reads two CSV files from the \texttt{public/data/} folder.
To update room information:

\begin{enumerate}
  \item \textbf{\texttt{space\_division.csv}} --- Export an updated room
        list from Astra (or create one manually).  Required columns:
        \texttt{BUILDING}, \texttt{LEVEL}, \texttt{STUDIO}, \texttt{ROOM},
        \texttt{ASTRA OCCUPANCY}.
  \item \textbf{\texttt{combined\_spaces.csv}} --- Define zones (groups of
        rooms treated as one space).  Required columns:
        \texttt{combined\_id}, \texttt{members},
        \texttt{capacity\_override}, \texttt{mode}.
\end{enumerate}

Replace the files in \texttt{public/data/} and refresh the browser.  The
dashboard will load the new data automatically.

% ════════════════════════════════════════════════════════════════
\part*{For Technical Users}
\addcontentsline{toc}{section}{How-to Guide --- Technical Users}
% ════════════════════════════════════════════════════════════════

\section{Tech Stack}

\begin{tabular}{@{}lll@{}}
\toprule
\textbf{Layer} & \textbf{Technology} & \textbf{Version} \\
\midrule
Frontend       & React + TypeScript  & 19 / 5.9 \\
Build tool     & Vite                & 7.2 \\
UI components  & Material-UI (MUI)   & v7.3 \\
Data tables    & MUI X DataGrid      & v8.20 \\
Forms          & React Hook Form + Zod & 7.67 / 4.1 \\
CSV parsing    & Papa Parse           & 5.5.3 \\
Styling        & Emotion             & --- \\
Linting        & ESLint + TS ESLint  & 9.39 \\
\bottomrule
\end{tabular}

% ────────────────────────────────────────────────────────────────
\section{Project Structure}

\begin{lstlisting}
Space_Allocation/
  public/
    data/
      space_division.csv      # Room inventory (Astra export)
      combined_spaces.csv     # Zone definitions
  src/
    components/
      AllocationForm.tsx      # Parameter entry form
      GroupTable.tsx           # Studios data grid
      RoomTable.tsx            # Room assignments data grid
      FloorSummary.tsx         # Floor capacity utilisation
      DiagnosticsPanel.tsx     # Warnings and errors
      FinanceSummaryPanel.tsx  # Staffing cost breakdown
      AllocationActions.tsx    # Rotate / Export buttons
      RoomSelectionPanel.tsx   # Toggle rooms in/out
    hooks/
      useSpaceData.ts          # CSV loading and room state
      useAllocationEngine.ts   # Grouping to allocation pipeline
    pages/
      DashboardPage.tsx        # Main layout
    utils/
      allocation.ts            # 3-pass allocation algorithm
      grouping.ts              # Studio generation
      capacity.ts              # Floor buffer tracking
      spaceTransform.ts        # CSV normalisation and floor summaries
      finance.ts               # Staffing cost calculations
      export.ts                # CSV export
      csv.ts                   # Papa Parse wrapper
    types/
      index.ts                 # TypeScript type definitions
    App.tsx                    # Theme setup + root component
    main.tsx                   # Application entry point
  docs/
    how-to-guide.tex           # This document
  package.json
  vite.config.ts
  tsconfig.json
  eslint.config.js
\end{lstlisting}

% ────────────────────────────────────────────────────────────────
\section{Architecture and Data Flow}

\begin{enumerate}
  \item \textbf{CSV Loading} ---
        \texttt{useSpaceData} fetches \texttt{space\_division.csv} and
        \texttt{combined\_spaces.csv} at startup, parses them with Papa
        Parse, and normalises the data into \texttt{Room[]} and
        \texttt{Floor[]} arrays.
  \item \textbf{User Input} ---
        \texttt{AllocationForm} collects program sizes, studio cap, mixing
        preference, and financial parameters.  Validation uses Zod.
  \item \textbf{Grouping} ---
        \texttt{generateStudios()} builds studio groups respecting the cap
        and mixing rules.
  \item \textbf{Allocation} ---
        \texttt{allocateStudiosToRooms()} runs the three-pass algorithm
        (see next section).
  \item \textbf{Finance} ---
        \texttt{buildFinanceSummary()} calculates staffing costs.
  \item \textbf{Display} ---
        MUI DataGrids render studios, rooms, floor states, diagnostics,
        and the finance breakdown.
\end{enumerate}

% ────────────────────────────────────────────────────────────────
\section{CSV Input Formats}
% ────────────────────────────────────────────────────────────────

\subsection{\texttt{space\_division.csv}}

Exported from Astra scheduling software.  Building and level values carry
forward to subsequent rows when left blank.

\begin{lstlisting}
BUILDING,LEVEL,STUDIO,ROOM,ASTRA OCCUPANCY,...
Design South (DS),Level - 1,-,143,30,,,,
,Level - 2,-,220,16,,,,
,,,221,24,,,,
\end{lstlisting}

\begin{tabular}{@{}llp{8cm}@{}}
\toprule
\textbf{Column} & \textbf{Required} & \textbf{Description} \\
\midrule
\texttt{BUILDING}          & Yes & Building name (carries forward when blank) \\
\texttt{LEVEL}             & Yes & Floor label (carries forward when blank) \\
\texttt{STUDIO}            & No  & If set, room is part of a combined space \\
\texttt{ROOM}              & Yes & Unique room number \\
\texttt{ASTRA OCCUPANCY}   & Yes & Room capacity (seats) \\
Others                     & No  & Ignored by the application \\
\bottomrule
\end{tabular}

\subsection{\texttt{combined\_spaces.csv}}

Defines zones --- groups of rooms that function as a single combined space.

\begin{lstlisting}
combined_id,members,capacity_override,mode
DS-320-321,"320,321",110,zone
DN-265-283,"265,267,269,271,277,281,283",210,zone
\end{lstlisting}

\begin{tabular}{@{}llp{8cm}@{}}
\toprule
\textbf{Column} & \textbf{Required} & \textbf{Description} \\
\midrule
\texttt{combined\_id}        & Yes & Unique identifier for the zone \\
\texttt{members}             & Yes & Comma-separated list of room numbers \\
\texttt{capacity\_override}  & No  & Override summed capacity; if blank, members are added \\
\texttt{mode}                & No  & Currently only \texttt{zone} is used \\
\bottomrule
\end{tabular}

% ────────────────────────────────────────────────────────────────
\section{Allocation Algorithm Deep Dive}
% ────────────────────────────────────────────────────────────────

The allocation engine (\texttt{src/utils/allocation.ts}) runs a
\textbf{greedy three-pass} algorithm.

\subsection{Preparation}

\begin{enumerate}
  \item \texttt{generateStudios()} creates studio groups from program
        sizes, respecting the studio cap and mixing preferences.
  \item Studios are sorted \emph{largest-first} so big groups are placed
        before small ones.
  \item Rooms are shuffled using a seeded linear-congruential generator
        (LCG) so that results are reproducible but can be rotated.
\end{enumerate}

\subsection{Three Allocation Passes}

\begin{tabular}{@{}clp{7.5cm}@{}}
\toprule
\textbf{Pass} & \textbf{Strategy} & \textbf{Room Constraint} \\
\midrule
1 & Strict &
  Only assign if the studio fits within the room's base capacity minus
  already-used seats: \newline
  \texttt{baseCapacity - usedCapacity >= studioSize} \\
2 & Next fit &
  Use the room's dynamic capacity (base + previously borrowed extra):
  \newline
  \texttt{dynamicCapacity - usedCapacity >= studioSize} \\
3 & Dynamic &
  Allow exceeding base capacity by borrowing from the floor buffer, if
  buffer remains: \newline
  \texttt{extraAllowed - extraUsed >= incrementalExtra} \\
\bottomrule
\end{tabular}

\subsection{Floor Buffer}

Each floor may exceed its summed base capacity by up to
\texttt{FLOOR\_BUFFER\_RATIO} (default 15\%).  When a room borrows extra
seats, the increment is deducted from the floor's remaining buffer.

\subsection{Diagnostics}

Any studio that fails all three passes is marked unassigned, and a
diagnostic message is recorded.  Common diagnostics include:

\begin{itemize}
  \item ``Unable to place S-001 (size 20).  Marked as unassignable.''
  \item ``No floor buffer left for Room 220.''
  \item ``Floor context missing for Room 143.''
\end{itemize}

% ────────────────────────────────────────────────────────────────
\section{Finance Model}
% ────────────────────────────────────────────────────────────────

The finance module (\texttt{src/utils/finance.ts}) calculates annualised
staffing costs using the following formula per role:

\begin{lstlisting}
compensation = baseRate * headCount
ERE          = compensation * ereRate
risk         = compensation * 1.1%
techFee      = compensation * 2.5%
subtotal     = compensation + ERE + risk + techFee
adminCharge  = subtotal * 8.5%
totalCost    = subtotal + adminCharge
\end{lstlisting}

Default rates:

\begin{tabular}{@{}lll@{}}
\toprule
\textbf{Role} & \textbf{Base Compensation} & \textbf{ERE Rate} \\
\midrule
Faculty  & \$85,000 / year                     & 30.6\% \\
TA / FA  & User-entered per semester, annualised & 11.0\% \\
Grader   & \$18,000 / year                      & 1.9\%  \\
\bottomrule
\end{tabular}

% ────────────────────────────────────────────────────────────────
\section{Customization Reference}
% ────────────────────────────────────────────────────────────────

\subsection{No-Code Customizations}

\begin{tabular}{@{}p{4cm}p{10cm}@{}}
\toprule
\textbf{What to Change} & \textbf{How} \\
\midrule
Room inventory &
  Replace \texttt{public/data/space\_division.csv} with a new Astra
  export. \\
Combined zones &
  Edit \texttt{public/data/combined\_spaces.csv}. \\
Program count, sizes, mixing &
  Change values in the dashboard form. \\
Include/exclude rooms &
  Toggle checkboxes in the Room Selection panel. \\
Shuffle assignment order &
  Click \emph{Rotate Allocation}. \\
\bottomrule
\end{tabular}

\subsection{Code-Level Customizations}

\begin{longtable}{@{}p{3.5cm}p{4.5cm}p{6cm}@{}}
\toprule
\textbf{What to Change} & \textbf{File} & \textbf{Constant / Location} \\
\midrule
\endfirsthead
\toprule
\textbf{What to Change} & \textbf{File} & \textbf{Constant / Location} \\
\midrule
\endhead
Floor buffer \% &
  \texttt{spaceTransform.ts} &
  \texttt{FLOOR\_BUFFER\_RATIO} (default \texttt{0.15}) \\
Default shuffle seed &
  \texttt{allocation.ts} &
  \texttt{DEFAULT\_SEED} (default \texttt{17}) \\
Theme colours &
  \texttt{App.tsx} &
  \texttt{createTheme(\{ palette \})} \\
Form defaults &
  \texttt{AllocationForm.tsx} &
  \texttt{defaultValues} object \\
Faculty salary &
  \texttt{finance.ts} &
  \texttt{COMPENSATION\_MATRIX[0].compensation} \\
Grader salary &
  \texttt{finance.ts} &
  \texttt{COMPENSATION\_MATRIX[2].compensation} \\
Rate constants &
  \texttt{finance.ts} &
  \texttt{RISK\_RATE}, \texttt{TECH\_FEE\_RATE}, etc. \\
Allocation logic &
  \texttt{allocation.ts} &
  \texttt{findRoomByStrategy()} \\
Grouping logic &
  \texttt{grouping.ts} &
  \texttt{generateStudios()} \\
Export columns &
  \texttt{export.ts} &
  \texttt{exportAllocationCsv()} \\
\bottomrule
\end{longtable}

% ────────────────────────────────────────────────────────────────
\section{Running the App}
% ────────────────────────────────────────────────────────────────

\begin{lstlisting}[language=bash]
# Install dependencies (re-run if package.json changes)
npm install

# Start development server at http://localhost:5173
npm run dev

# Type-check and build for production (output in /dist)
npm run build

# Serve the production build locally
npm run preview

# Lint the codebase
npm run lint
\end{lstlisting}

% ────────────────────────────────────────────────────────────────
\section{Notes and Next Steps}
% ────────────────────────────────────────────────────────────────

\begin{itemize}
  \item The dev server is not auto-started; run \texttt{npm run dev} when
        you are ready.
  \item Vite warns about a ${>}500$\,kB bundle because of MUI/DataGrid.
        Add manual chunking in \texttt{vite.config.ts} if you need
        smaller bundles.
  \item If facilities provide explicit floor-area or capacity-limit
        columns, replace the proxy logic in \texttt{spaceTransform.ts} to
        use those authoritative values.
  \item No test framework is configured yet.  Consider adding Vitest for
        unit tests covering \texttt{allocation.ts} and
        \texttt{grouping.ts}.
\end{itemize}

\end{document}
